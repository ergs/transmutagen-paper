\todo{Anthony will need to review this text, or rewrite it if necessary.}
In addition to numeric tests, it is possible to run a sanity check on the CRAM
method based on physical tests. If $A$ is a transmutation matrix with no
fission, $e^{-At}$ represents {\color{red}\ldots}. The $i$th column of
$e^{-At}=e^{-At}I$ is equal to $e^{-At}\delta_{i}$ where $\delta_i$ is a
column vector with $1$ in the $i$th row and $0$ elsewhere. That is, the $i$th
column of $e^{-At}$ represents the transmutation of a unit mass of the $i$th
nuclide after $t$ seconds with no fission. By the conservation of mass, these
columns should each sum to 1, the mass that was started. While pure
transmutation of an isolated unit mass of each nuclide may not be physically
feasible, this provides a mathematical sanity check for different methods of
computing $e^{-At}$.
