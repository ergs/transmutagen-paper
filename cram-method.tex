Consider the function $e^{-t}$ on the interval $[0,\infty)$. Let
$\pi_{k,k}\subset \mathbb{R}(t)$ be the set of rational functions $r_{k,k}(t)
= p_k(t)/q_k(t)$, where $p_k(t), q_k(t)\in \mathbb{R}[t]$ are polynomials of
degree at most $k$.

Chebyshev Rational Approximation Method (CRAM) approximation of degree $k$
for $e^{-t}$ is the unique rational function $\hat{r}_{k,k}(t)$ such that
\begin{equation}
  \sup_{t\in[0, \infty)}|\hat{r}_{k, k}(t) - e^{-t}|
\end{equation}
is minimized. That is,
\begin{equation}
  \epsilon_{k,k} = \sup_{t\in[0, \infty)}|\hat{r}_{k, k}(t) - e^{-t}| = \inf_{r_{k,k}\in\pi_{k,k}}\left\{\sup_{t\in[0, \infty)}|r_{k, k}(t) - e^{-t}|\right\}.
\end{equation}
$\epsilon_{k,k}$ is the absolute error of the approximation. It has been shown
that $\epsilon_{k,k} = O(H^{-k})$, where $H=9.289\ldots$.

Transmutagen implements the Remez algorithm to compute the CRAM approximation.
All computations are done using SymPy symbolic expressions with arbitrary
precision floating point numbers, which are backed by the mpmath library.

The interval $[0, \infty)$ is first translated to $[-1, 1)$ via the
transformation $t\mapsto c\frac{t+1}{t-1}$.
% cite Carpenter paper here
% TODO: discuss this *after* discussing finding the roots.
The constant $c$ is chosen so that the minimax roots of the approximation are
distributed sufficiently evenly across the interval $[-1, 1)$. We found that
the value $c=0.6k$ works well.

The expressions
\begin{equation}
  r := \frac{p_kt^k + \cdots + p_1t + p_0}{q_kt^k + \cdots +
    q_1t + 1},
\end{equation}
\begin{equation}
  D := e^{c\frac{t+1}{t-1}} - r,
\end{equation}
and
\begin{equation}
  E := E + (-1)^i\epsilon
\end{equation}
are represented symbolically as a SymPy expression. This expression $E$ is then
evaluated in $t$ at $k + 2$ points in the interval $[-1, 1)$ for $i=0\ldots
k+1$.
% TODO: Chebyshev nodes
This results in a nonlinear system of $k+2$ equations in $k+2$ variables
($p_0,\ldots,p_k,q_1,\ldots,q_k,\epsilon$). These are solved using SymPy's
\texttt{nsolve} function, which internally uses \texttt{mpmath.findroot}. This
uses Newton's method, with a symbolically computed Jacobian.

The solution to this system is then substituted into $D$. The critical points
of this function on the interval $[-1, 1)$ are then found. This is done by
taking the symbolic derivative of $D$, splitting the interval into
subintervals, and performing bisection using \texttt{nsolve} on each
subinterval. Call these points, along with the end-points $D|_{t=-1}$ and
$\lim_{t\to 1} D=-r|_{t=1}$, $z_i$. There are $2(k+1)$ points $z_i$. The
points $z_i$
