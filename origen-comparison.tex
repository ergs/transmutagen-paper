\begin{itemize}
\item \it{Mass fraction vs. atom fraction}
\item {\it Same data}
\item $\alpha$ as He4
\item Fission yields for actinides
\item Cf248 not produced in 1 Month brwu, brwus, pwrdu3th, pwrputh, pwru,
  pwru50, pwrue, pwrus
\item More differences for larger time steps than 100 days
\item The C solver and SuperLU agree to within 1e-12
\item UMFPACK agrees to within 1e-8
\item \todo{Anthony will need to write this. See ORIGEN run logs in Google Drive}
\end{itemize}

As a more rigorous sanity check, the ORIGEN 2.2~\cite{ationneeded} library was
compared against for several time steps and starting nuclides. ORIGEN 2.2 was
run against 22 ORIGEN libraries (\texttt{amo0tttr.\allowbreak{}lib},
\texttt{amo1ttta.\allowbreak{}lib}, \texttt{amo1tttc.\allowbreak{}lib},
\texttt{amo2ttta.\allowbreak{}lib}, \texttt{amo2tttr.\allowbreak{}lib},
\texttt{amopttta.\allowbreak{}lib}, \texttt{amoptttc.\allowbreak{}lib},
\texttt{amoptttr.\allowbreak{}lib}, \texttt{amopuutc.\allowbreak{}lib},
\texttt{amopuuua.\allowbreak{}lib}, \texttt{amopuuuc.\allowbreak{}lib},
\texttt{amopuuur.\allowbreak{}lib}, \texttt{amoruuua.\allowbreak{}lib},
\texttt{amoruuur.\allowbreak{}lib}, \texttt{bwru.\allowbreak{}lib},
\texttt{bwrus.\allowbreak{}lib}, \texttt{canduseu.\allowbreak{}lib},
\texttt{crbra.\allowbreak{}lib}, \texttt{crbrc.\allowbreak{}lib},
\texttt{crbrr.\allowbreak{}lib}, \texttt{emopuuuc.\allowbreak{}lib},
\texttt{fftfc.\allowbreak{}lib}, \texttt{pwrdu3th.\allowbreak{}lib},
\texttt{pwrputh.\allowbreak{}lib}, \texttt{pwru.\allowbreak{}lib},
\texttt{pwru50.\allowbreak{}lib}, \texttt{pwrue.\allowbreak{}lib}, and
\texttt{pwrus.\allowbreak{}lib}), 7 time steps ($t= 1$ second, 1 day, 1 month,
1 year, 10 years, 1000 years, and 1 million years), and 7 starting nuclides
(Th232, U233, U235, U238, Pu239, Pu241, Cm245, Cf249), and the results were
compared against CRAM using \texttt{sympy.\allowbreak{}lambdify} using
UMFPACK, \texttt{sympy.\allowbreak{}lambdify} using SuperLU, and a C solver
generated by transmutagen against the ORIGEN libraries' combined sparsity
pattern.

\todo{This is just describing Meeseeks for now.} The speed of these analyses is summarized in
Figures~\ref{fig:origen-scopatz},~\ref{fig:origen-aaron},
and~\ref{fig:origen-meeseeks}. The performance of ORIGEN is dependent on the
time step being computed. For small time steps (1 second to 1 year) ORIGEN
takes from 0.5\;s to 1\;s to run. At larger time steps, it takes increasingly
long---as long as 1110\;s at 1 million years. CRAM on the other hand, has a
runtime that is independent of the timestep. UMFPACK runs from 130\;ms to
280\;ms, and SuperLU runs from 80\;ms to 150\;ms. The generated C solver runs
from 30\;ms to 50\;ms. \todo{Should we report means or anything like that?}

In addition to performance comparison, the runs were also used to compare the
output of the four solvers. A large number of discrepancies were found for the
larger timesteps. However, according to the ORIGEN manual, ORIGEN is only
designed to give accurate values for time steps less than 100
days~\cite{ationneeded}.

All comparisons against the computed values were done with an absolute
tolerance of $10^{-5}$ and a relative tolerance of $10^{-3}$.\footnote{Using
  \texttt{numpy.\allowbreak{}allclose}, which compares $|actual -
  desired|$ to $atol + rtol|desired|$,
  where $actual$ is the value computed by CRAM and $desired$ is the
  value computed by ORIGEN.} The relative tolerance was chosen because ORIGEN
produces 4 digits of output. The absolute tolerance was chosen ORIGEN computes
the matrix exponential $e^{-A}$ using a Taylor expansion method using the error term
$\frac{e^{\mathrm{ASUM}}\mathrm{ASUM}^n}{n!}$,\footnote{ORIGEN computes this using
    Sterling's approximation, i.e.,
    $\frac{e^{\mathrm{ASUM}}(\frac{\mathrm{ASUM}e}{n})^n}{\sqrt{2\pi
      n}}$. c.f. lines 5075-5100 of the ORIGEN 2.2 source code.} where
$\mathrm{ASUM}$ is the maximum of the column sums of the matrix $A$ and $n =
3.5\mathrm{ASUM} + 6$. Due to fission,\todo{Write more here} the maximum
column sum is $\approx 2$, giving $\frac{e^{2}2^{13}}{13!}\approx 10^{-5}$.

For the 1 second, 1 day, and 1 month time steps,


\todo{Which of Figures~\ref{fig:origen-scopatz},~\ref{fig:origen-aaron},~\ref{fig:origen-meeseeks} should we show here?}

\todo{Pull in exact max/min data}
\begin{figure}[!ht]
\centering
\resizebox{0.9\textwidth}{!}{\input{origen-scopatz.pgf}}
\caption{Runtimes for different solvers computing transmutation over several starting libraries, nuclides, and timesteps.
(Scopatz machine)}
\label{fig:origen-scopatz}
\end{figure}

% Aaron's Mac: Mid 2012 MacbookPro, 2.3 GHz Intel Core i7, 8 GB 1600 MHz DDR3
\begin{figure}[!ht]
\centering
\resizebox{0.9\textwidth}{!}{\input{origen-aaron.pgf}}
\caption{Runtimes for different solvers computing transmutation over several starting libraries, nuclides, and timesteps.
 (Aaron machine)}
\label{fig:origen-aaron}
\end{figure}

\begin{figure}[!ht]
\centering
\resizebox{0.9\textwidth}{!}{\input{origen-meeseeks.pgf}}
\caption{Runtimes for different solvers computing transmutation over several starting libraries, nuclides, and timesteps.
 (Meeseeks)}
\label{fig:origen-meeseeks}
\end{figure}
