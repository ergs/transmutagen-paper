\begin{itemize}
\item \it{Mass fraction vs. atom fraction}
\item {\it Same data}
\item $\alpha$ as He4
\item Fission yields for actinides
\item Cf248 not produced in 1 Month brwus
\item More differences for larger time steps than 100 years
\item \todo{Anthony will need to write this. See ORIGEN run logs in Google Drive}
\end{itemize}

As a more rigorous sanity check, the ORIGEN 2.2~\cite{ationneeded} library was
compared against, for several time steps and starting nuclides. ORIGEN 2.2 was
run against 22 ORIGEN libraries \todo{list the libraries?}, 7 time steps
($t= 1$ second, 1 day, 1 month, 1 year, 10 years, 1000 years, and 1 million
years), and 7 starting nuclides (Th232, U233, U235, U238, Pu239, Pu241, Cm245,
Cf249), and the results were compared against CRAM using
\texttt{sympy.\allowbreak{}lambdify} using UMFPACK,
\texttt{sympy.\allowbreak{}lambdify} using SuperLU, and a C solver generated
by transmutagen against the ORIGEN libraries' sparsity pattern.

\todo{This is just describing Meeseeks for now.} The speed of these analyses is summarized in
figures~\ref{fig:origen-scopatz},~\ref{fig:origen-aaron},
and~\ref{fig:origen-meeseeks}. The performance of ORIGEN is dependent on the
time step being computed. For small time steps (1 second to 1 year) ORIGEN
takes from 0.5\;s to 1\;s to run. At larger time steps, it takes increasingly
long---as long as 1110\;s at 1 million years. CRAM on the other hand, has a
runtime that is independent of the timestep. UMFPACK runs from 130\;ms to
280\;ms, and SuperLU runs from 80\;ms to 150\;ms. The generated C solver runs
from 30\;ms to 50\;ms. \todo{Should we report means or anything like that?}

In addition to performance comparison, the runs were also used to compare the
output of the four solvers.

\it{From transmutagen's \texttt{compute\_mismatch}:}
The default atol is 1e-5 because ORIGEN stops the taylor expansion with
the error term exp(ASUM)*ASUM**n/n! (using Sterling's approximation),
where n = 3.5*ASUM + 6 and ASUM is the max of the column sums. The max of
the column sums is ~2 because of fission, giving ~1e-5 (see ORIGEN lines
5075-5100)

\todo{Which of Figures~\ref{fig:origen-scopatz},~\ref{fig:origen-aaron},~\ref{fig:origen-meeseeks} should we show here?}

\todo{Pull in exact max/min data}
\begin{figure}[!ht]
\centering
\resizebox{0.9\textwidth}{!}{\input{origen-scopatz.pgf}}
\caption{Runtimes for different solvers computing transmutation over several starting libraries, nuclides, and timesteps.
(Scopatz machine)}
\label{fig:origen-scopatz}
\end{figure}

\begin{figure}[!ht]
\centering
\resizebox{0.9\textwidth}{!}{\input{origen-aaron.pgf}}
\caption{Runtimes for different solvers computing transmutation over several starting libraries, nuclides, and timesteps.
 (Aaron machine)}
\label{fig:origen-aaron}
\end{figure}

\begin{figure}[!ht]
\centering
\resizebox{0.9\textwidth}{!}{\input{origen-meeseeks.pgf}}
\caption{Runtimes for different solvers computing transmutation over several starting libraries, nuclides, and timesteps.
 (Meeseeks)}
\label{fig:origen-meeseeks}
\end{figure}
