Once the CRAM approximation of degree $k$, $\hat{r}_{k, k}(t)$ has been
computed, it can be used to compute the matrix exponential $e^{-A}$ by
substituting the matrix $A$ for $t$ in the expression. Transmutagen has a
Python class, \texttt{MatrixNumPyPrinter}, which subclasses and extends the
\texttt{sympy.printing.lambdarepr.NumPyPrinter} class, which is used by
\texttt{sympy.lambdify} to convert a SymPy expression to a string representing
an equivalent NumPy/SciPy expression. The \texttt{NumPyPrinter} class by
default deals with scalar expressions vectorized over arrays, and the
\texttt{MatrixNumPyPrinter} class extends it so that the expression operates
over matrices. In particular, the class ensures that

\begin{itemize}
\item multiplications use scalar multiplication (\texttt{*}) or
  matrix multiplication (\texttt{@}), as appropriate,
\item addition of a variable and a constant, such as $t + 2$, is represented
  by the addition of an identity matrix, $t + 2I$. Since the shape of the
  matrix is not known ahead of time, Transmutagen uses a special
  \texttt{autoeye} class which automatically evaluates as \texttt{scipy.sparse.linalg.eye} at runtime when the shape is known.
\item Divisions are represented by matrix solves, e.g., $\frac{t + 1}{t - 1}$
  would be represented by \texttt{solve(t - autoeye(1), t + autoeye(1))}.
\end{itemize}

The naive method of computing the matrix exponential via $\hat{r}_{k,
  k}(t)=\frac{p_kt^k + \cdots + p_0}{q_kt^k + \cdots q_1t + 1}$
is to compute $(q_kA^k + \cdots + q_1A + I)\backslash(p_kA^k + \cdots + I)$, where
$\backslash$ is a matrix solve. However, this method is very unstable, as the
coefficients $p_i,q_i$ are quite small (the smallest on the order of
$10^{-2k}$), and taking powers of $A$ is also unstable.

A better method is to perform a partial fraction decomposition on $\hat{r}_{k,
  k}(t)$:
\begin{equation}
  \hat{r}_{k, k}(t) = \alpha_0 + \sum_{i=1}^k \frac{\alpha_i}{t - \theta_i},
\end{equation}
where $\theta_i$ are the roots of $q_kt^k + \cdots + q_1t + 1$, which are all
nonreal and have multiplicity 1, $\alpha_i$ is the residue of $\hat{r}_{k,
  k}(t)$ at $t=\theta_i$, and $\alpha_0$ is the residue at infinity. Since the
roots $\theta_i$ all come in complex conjugate pairs, the sum can be reduced to
\begin{equation}
  \hat{r}_{k, k}(t) = \alpha_0 + \mathrm{Re}\left(\sum_{i=1}^{k/2} \frac{\alpha_i}{t - \theta_i}\right),
\end{equation}
which is computed on a matrix $A$ via
\begin{equation}
  \hat{r}_{k, k}(A) = \alpha_0I + \mathrm{Re}\left(\sum_{i=1}^{k/2} (A -
    \theta_i I)\backslash(\alpha_i I) \right).
\end{equation}

The roots $\theta_i$ were computed with \texttt{sympy.nsolve} using Newton's
method.
% TODO: Some discussion on the number of digits needed for the roots here.

We attempted expand the real part of this expression via
\begin{equation}
\mathrm{Re}\left(\frac{\alpha_i}{t - \theta_i}\right) = \frac{\mathrm{Re}{(\alpha_{i})}t - \mathrm{Re}{(\alpha_{i})} \mathrm{Re}{(\theta_{i})} - \mathrm{Im}{(\alpha_{i})} \mathrm{Im}{(\theta_{i})}}{\left(t - \mathrm{Re}{(\theta_{i})}\right)^{2} + \mathrm{Im}{(\theta_{i})}^{2}},
\end{equation}
and compute
\begin{equation}
\left(\left(A - \mathrm{Re}{(\theta_{i})I}\right)^{2} +
  \mathrm{Im}{(\theta_{i})}^{2}I\right)\backslash \left(\mathrm{Re}{(\alpha_{i})}A - (\mathrm{Re}{(\alpha_{i})} \mathrm{Re}{(\theta_{i})} + \mathrm{Im}{(\alpha_{i})} \mathrm{Im}{(\theta_{i})})I\right).
\end{equation}
However, this was found to be numerically unstable, most likely because of the
additional matrix power. % TODO: verify this
