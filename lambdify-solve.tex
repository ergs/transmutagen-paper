Once the CRAM approximation of degree $k$, $\hat{r}_{k, k}(t)$ has been
computed, it can be used to compute the matrix exponential $e^{-A}b$ by
substituting the matrix $A$ for $t$ in the expression. Transmutagen has a
Python class, \texttt{MatrixNumPyPrinter}, which subclasses and extends the
\texttt{sympy.printing.lambdarepr.NumPyPrinter} class, which is used by
\texttt{sympy.lambdify} to convert a SymPy expression to a string representing
an equivalent NumPy/SciPy~\cite{ationneeded} expression. The
\texttt{NumPyPrinter} class by default deals with scalar expressions
vectorized over arrays, and the \texttt{MatrixNumPyPrinter} class extends it
so that the expression operates over matrices. In particular, the class
ensures that

\begin{itemize}
\item multiplications use scalar multiplication (\texttt{*}) or
  matrix multiplication (\texttt{@}), as appropriate,
\item addition of a variable and a constant, such as $t + 2$, is represented
  by the addition of an identity matrix, $A + 2I$. Since the shape of the
  matrix is not known ahead of time, Transmutagen uses a special
  \texttt{autoeye} class which automatically evaluates as \texttt{scipy.sparse.linalg.eye} at runtime when the shape is known.
\item Divisions are represented by matrix solves, e.g., $\frac{2}{t - 1}b$
  would be represented by \texttt{solve(A - autoeye(1), 2b)}.
\end{itemize}
