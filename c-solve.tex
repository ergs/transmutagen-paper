The CRAM method involves solving
\begin{equation}
 (A - \theta I)\backslash(\alpha b)
\end{equation}
$k/2$ times for fixed $A$ and $b$ and varying $\theta$ and $\alpha$. For the
transmutation problem the $A$ matrix is a sparse matrix with a fixed sparsity
pattern: every row/column in $A$ corresponds to a transmutation of one nuclide
to another, and only certain transmutations are ever physically possible.

The LU solve method for solving a $Ax=b$ without pivoting is, in pseudocode,

\begin{verbatim}
# First, decompose A
# The lower triangular part of LU will be L - I and the upper triangular part will
# be U.

LU = copy(A)

for i=1..N:
    for j=i..N:
        LU[j, i] /= LU[i, i]

for k=i..N:
    LU[j, k] -= LU[j, i]*LU[i, k]

# Now perform the solve
x = copy(b)

# Forward substitution
for i=1..N:
    for j=1..i:
        x[i] -= LU[i, j]*x[j]

# Backward substitution
for i=N..1:
    for j=i..N:
        x[i] -= LU[i, j]*x[j]
    x[i] /= LU[i, i]
\end{verbatim}

Critically, it is known ahead of time which entries of $A$ are nonzero. Thus,
the steps above can be reduced or eliminated for the entries that are known to
be zero. Furthermore, for the transmutation problem, $A$ consists only of real
entries, whereas for $A - \theta I$, $\theta$ is a nonreal complex number. So
the diagonal entries are not zero. The roots of the denominators of the CRAM
approximations do not correspond to the eigenvalues of the transmutation
matrix, so the solve is never degenerate.

Transmutagen has a function that takes a given sparsity pattern for $A$ and
generates a C function that solves $(A - \theta)x =\alpha b$. Additional C
functions are generated from the CRAM approximations of given orders (by
default, 6, 8, 10, 12, 14, 16, and 18), which compute $e^{-A}b$. The sparsity
pattern is generalized as a list of nuclides and a list of ``from/to'' pairs
of nuclides whose transmutation may be represented in the input matrix $A$.
