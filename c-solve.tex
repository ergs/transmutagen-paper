The command \texttt{python -m transmutagen.gensolve} takes a given sparsity
pattern for $A$ and generates a C function that solves
$(A - \theta)x =\alpha b$ (a default sparsity pattern based on data from
PyNE~\cite{ationneeded} is included with Transmutagen\todo{Discuss separate sparsity
  pattern for ORIGEN nuclides}). Additional C functions
are generated from the CRAM approximations of given orders (by default, 6, 8,
10, 12, 14, 16, and 18, but any even order can be used), which compute
$e^{-A}b$.

The command generates a C source file and header from a
Jinja~\cite{ationneeded} template based on the pseudocode in
Algorithm~\ref{alg:lu-pseudocode}. The source uses C99 complex numbers for the
arithemetic. Each line that is known to be a no-op from the provided sparsity
pattern is automatically removed. The resulting C source file is 12 MB with
the default orders \todo{Update this for latest sparsity pattern}. Compilation
of this file requires disabling most optimizations, as otherwise the compiler
either does not finish or runs out of memory. However, certain compilation
flags were found to speed up the performance of the algorithm, particularly
flags to speed up complex number operations. By default, complex numbers in C
are slow due to NaN checks, but these can be disabled to make the code
faster. \todo{Better justification on why these can be disabled.} Through
experimentation, we found the GCC flags \texttt{-O0 -fcx-fortran-rules
  -fcx-limited-range -ftree-sra -ftree-ter -fexpensive-optimizations} provided
speedups without adversely slowing down compile times. For Clang, we found the
flags \texttt{-O0 --ffast-math}. \todo{Should we discuss Clang here, such as
  how it is slower and doesn't seem to optimize complex numbers?}
\todo{Some performance information here (compile times; timing with/without
  optimizations). See also Section~\ref{sec:origen-speed}.}

To make the method more accessible to nuclear scientists, the sparsity pattern
input file is formatted as a JSON file with a list of nuclides in the order
they appear in the matrix, and a list of from--to pairs of transitions between
nuclides that may be represented in the input matrix $A$. The performance of
this method is outlined in Section~\ref{sec:origen-speed}.
