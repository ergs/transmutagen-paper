The CRAM method using equation~\ref{eq:part-frac-matrix-re} involves solving
\begin{equation}
 (A - \theta I)\backslash(\alpha b)
\end{equation}
$k/2$ times for fixed $A$ and $b$ and varying $\theta$ and $\alpha$. For the
transmutation problem the $A$ matrix is a sparse matrix with a fixed sparsity
pattern: every row/column in $A$ corresponds to a transmutation of one nuclide
to another, and only certain transmutations are ever physically possible.

The LU solve method for solving a $Ax=b$ without pivoting is shown in
pseudocode in Figure~\ref{lu-pseudocode}.
\begin{figure}[h]
  \caption{LU solve of $Ax=b$ without pivoting.}\label{lu-pseudocode}
  \begin{algorithmic}
  \STATE \COMMENT{First, decompose $A$.}
  \STATE \COMMENT{The lower triangular part of $LU$ will be $L - I$ and the upper triangular part will be $U$.}
  \STATE
  \STATE $LU \leftarrow \mathrm{copy}(A)$
  \STATE
  \FOR{$i=1$ \TO $N$}
    \FOR{$j=i$ \TO $N$}
        \STATE $LU_{j, i} \leftarrow LU_{j, i}/LU_{i, i}$
    \ENDFOR
  \ENDFOR
  \STATE
  \FOR{$k=i$ \TO $N$}
    \STATE $LU_{j, k} \leftarrow LU_{j, k} - LU_{j, i}\cdot LU_{i, k}$
  \ENDFOR
  \STATE
  \STATE \COMMENT{Now perform the solve.}
  \STATE $x \leftarrow \mathrm{copy}(b)$
  \STATE
  \STATE \COMMENT{Forward substitution}
  \FOR{$i=1$ \TO $N$}
      \FOR{$j=1$ \TO $i$}
          \STATE $x_i \leftarrow x_i - LU_{i, j}\cdot x_j$
      \ENDFOR
  \ENDFOR
  \STATE
  \STATE \COMMENT{Backward substitution}
  \FOR{$i=N$ \TO $1$}
      \FOR{$j=i$ \TO $N$}
          \STATE $x_i \leftarrow x_i -LU_{i, j}\cdot x_j$
      \ENDFOR
      \STATE $x_i \leftarrow x_i/LU_{i, i}$
  \ENDFOR

\end{algorithmic}
\end{figure}

Critically, it is known ahead of time which entries of $A$ are nonzero. Thus,
the steps above can be reduced or eliminated for the entries that are known to
be zero. Furthermore, for the transmutation problem, $A$ consists only of real
entries, whereas for $A - \theta I$, $\theta$ is a nonreal complex number. So
the diagonal entries are not zero. The roots of the denominators of the CRAM
approximations do not correspond to the eigenvalues of the transmutation
matrix, so the solve is never degenerate.

The command \texttt{python -m transmutagen.gensolve} takes a given sparsity
pattern for $A$ and generates a C function that solves
$(A - \theta)x =\alpha b$ (a default sparsity pattern based on data from
PyNE~\cite{ationneeded} is included with Transmutagen). Additional C functions
are generated from the CRAM approximations of given orders (by default, 6, 8,
10, 12, 14, 16, and 18, but any even order can be used), which compute
$e^{-A}b$. To make the method more accessible to nuclear scientists, sparsity
pattern is generalized as a list of nuclides and a list of transitions between
nuclides (from--to pairs) that may be represented in the input matrix $A$. The
performance of this method is outlined in section~\ref{sec:origen-speed}.
