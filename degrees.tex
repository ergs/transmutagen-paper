In our tests, when using CRAM to compute transmutation with machine double
floating point arithmetic, the maximum absolute error was bounded at around
$10^{-14}$. Figure~\ref{fig:degrees} shows how the error levels off at around
degree 16 for a transmutation computation for smaller timesteps and around
degree 18 for larger timesteps. The error in general is higher for higher time
steps. This is because multiplying a matrix $A$ by $t$ scales the spectrum of
$A$ by $t$.

\begin{figure}[!ht]
\centering
\resizebox{0.9\textwidth}{!}{In our tests, when using CRAM to compute transmutation with machine double
floating point arithmetic, the maximum absolute error was bounded at around
$10^{-14}$. Figure~\ref{fig:degrees} shows how the error levels off at around
degree 16 for a transmutation computation for smaller timesteps and around
degree 18 for larger timesteps. The error in general is higher for higher time
steps. This is because multiplying a matrix $A$ by $t$ scales the spectrum of
$A$ by $t$.

\begin{figure}[!ht]
\centering
\resizebox{0.9\textwidth}{!}{In our tests, when using CRAM to compute transmutation with machine double
floating point arithmetic, the maximum absolute error was bounded at around
$10^{-14}$. Figure~\ref{fig:degrees} shows how the error levels off at around
degree 16 for a transmutation computation for smaller timesteps and around
degree 18 for larger timesteps. The error in general is higher for higher time
steps. This is because multiplying a matrix $A$ by $t$ scales the spectrum of
$A$ by $t$.

\begin{figure}[!ht]
\centering
\resizebox{0.9\textwidth}{!}{In our tests, when using CRAM to compute transmutation with machine double
floating point arithmetic, the maximum absolute error was bounded at around
$10^{-14}$. Figure~\ref{fig:degrees} shows how the error levels off at around
degree 16 for a transmutation computation for smaller timesteps and around
degree 18 for larger timesteps. The error in general is higher for higher time
steps. This is because multiplying a matrix $A$ by $t$ scales the spectrum of
$A$ by $t$.

\begin{figure}[!ht]
\centering
\resizebox{0.9\textwidth}{!}{\input{degrees.pgf}}
\caption{Maximum absolute difference of computed values for degree $n$
  compared to $n-2$ for varying degrees and timesteps, using PWRU50 data with
  an initial vector $b$ representing a unit mass of U235.}
\label{fig:degrees}
\end{figure}

Figure~\ref{fig:error-plot} shows the error of the CRAM approximation in the
complex plane (note that it shows $\hat{r}(-t)$, so that it can be compared
against the spectrum of $A$). Comparing it with
Figures~\ref{fig:eigenvals-pwru50} and~\ref{fig:eigenvals-decay}, it can be
seen that CRAM should provide an accurate approximation for transmutation
matrices, but not for decay matrices.

\begin{figure}[!ht]
\centering
\includegraphics[width=0.9\textwidth]{error-plot.pdf}
\caption{Error of the approximation, $\left |\hat{r}_{14,14}(-t) - e^{t}\right
  |$ in the complex plane. }
\label{fig:error-plot}
\end{figure}

\begin{figure}[!ht]
\centering
\includegraphics[width=0.9\textwidth]{eigenvals_pwru50.pdf}
\caption{Eigenvalues of the transmutation matrix for pwru50, as computed by
  \texttt{scipy.\allowbreak{}sparse.\allowbreak{}linalg.\allowbreak{}eigen.\allowbreak{}eigs()}
  \todo{Cite ORIGEN and explain what this is (a pressurized water reactor with
    burnup of 50 MWd/kg)}.}
\label{fig:eigenvals-pwru50}
\end{figure}

\begin{figure}[!ht]
\centering
\includegraphics[width=0.9\textwidth]{eigenvals_decay.pdf}
\caption{Eigenvalues of the decay matrix, as computed by
  \texttt{scipy.\allowbreak{}sparse.\allowbreak{}linalg.\allowbreak{}eigen.\allowbreak{}eigs()}.}
\label{fig:eigenvals-decay}
\end{figure}
}
\caption{Maximum absolute difference of computed values for degree $n$
  compared to $n-2$ for varying degrees and timesteps, using PWRU50 data with
  an initial vector $b$ representing a unit mass of U235.}
\label{fig:degrees}
\end{figure}

Figure~\ref{fig:error-plot} shows the error of the CRAM approximation in the
complex plane (note that it shows $\hat{r}(-t)$, so that it can be compared
against the spectrum of $A$). Comparing it with
Figures~\ref{fig:eigenvals-pwru50} and~\ref{fig:eigenvals-decay}, it can be
seen that CRAM should provide an accurate approximation for transmutation
matrices, but not for decay matrices.

\begin{figure}[!ht]
\centering
\includegraphics[width=0.9\textwidth]{error-plot.pdf}
\caption{Error of the approximation, $\left |\hat{r}_{14,14}(-t) - e^{t}\right
  |$ in the complex plane. }
\label{fig:error-plot}
\end{figure}

\begin{figure}[!ht]
\centering
\includegraphics[width=0.9\textwidth]{eigenvals_pwru50.pdf}
\caption{Eigenvalues of the transmutation matrix for pwru50, as computed by
  \texttt{scipy.\allowbreak{}sparse.\allowbreak{}linalg.\allowbreak{}eigen.\allowbreak{}eigs()}
  \todo{Cite ORIGEN and explain what this is (a pressurized water reactor with
    burnup of 50 MWd/kg)}.}
\label{fig:eigenvals-pwru50}
\end{figure}

\begin{figure}[!ht]
\centering
\includegraphics[width=0.9\textwidth]{eigenvals_decay.pdf}
\caption{Eigenvalues of the decay matrix, as computed by
  \texttt{scipy.\allowbreak{}sparse.\allowbreak{}linalg.\allowbreak{}eigen.\allowbreak{}eigs()}.}
\label{fig:eigenvals-decay}
\end{figure}
}
\caption{Maximum absolute difference of computed values for degree $n$
  compared to $n-2$ for varying degrees and timesteps, using PWRU50 data with
  an initial vector $b$ representing a unit mass of U235.}
\label{fig:degrees}
\end{figure}

Figure~\ref{fig:error-plot} shows the error of the CRAM approximation in the
complex plane (note that it shows $\hat{r}(-t)$, so that it can be compared
against the spectrum of $A$). Comparing it with
Figures~\ref{fig:eigenvals-pwru50} and~\ref{fig:eigenvals-decay}, it can be
seen that CRAM should provide an accurate approximation for transmutation
matrices, but not for decay matrices.

\begin{figure}[!ht]
\centering
\includegraphics[width=0.9\textwidth]{error-plot.pdf}
\caption{Error of the approximation, $\left |\hat{r}_{14,14}(-t) - e^{t}\right
  |$ in the complex plane. }
\label{fig:error-plot}
\end{figure}

\begin{figure}[!ht]
\centering
\includegraphics[width=0.9\textwidth]{eigenvals_pwru50.pdf}
\caption{Eigenvalues of the transmutation matrix for pwru50, as computed by
  \texttt{scipy.\allowbreak{}sparse.\allowbreak{}linalg.\allowbreak{}eigen.\allowbreak{}eigs()}
  \todo{Cite ORIGEN and explain what this is (a pressurized water reactor with
    burnup of 50 MWd/kg)}.}
\label{fig:eigenvals-pwru50}
\end{figure}

\begin{figure}[!ht]
\centering
\includegraphics[width=0.9\textwidth]{eigenvals_decay.pdf}
\caption{Eigenvalues of the decay matrix, as computed by
  \texttt{scipy.\allowbreak{}sparse.\allowbreak{}linalg.\allowbreak{}eigen.\allowbreak{}eigs()}.}
\label{fig:eigenvals-decay}
\end{figure}
}
\caption{Maximum absolute difference of computed values for degree $n$
  compared to $n-2$ for varying degrees and timesteps, using PWRU50 data with
  an initial vector $b$ representing a unit mass of U235.}
\label{fig:degrees}
\end{figure}

Figure~\ref{fig:error-plot} shows the error of the CRAM approximation in the
complex plane (note that it shows $\hat{r}(-t)$, so that it can be compared
against the spectrum of $A$). Comparing it with
Figures~\ref{fig:eigenvals-pwru50} and~\ref{fig:eigenvals-decay}, it can be
seen that CRAM should provide an accurate approximation for transmutation
matrices, but not for decay matrices.

\begin{figure}[!ht]
\centering
\includegraphics[width=0.9\textwidth]{error-plot.pdf}
\caption{Error of the approximation, $\left |\hat{r}_{14,14}(-t) - e^{t}\right
  |$ in the complex plane. }
\label{fig:error-plot}
\end{figure}

\begin{figure}[!ht]
\centering
\includegraphics[width=0.9\textwidth]{eigenvals_pwru50.pdf}
\caption{Eigenvalues of the transmutation matrix for pwru50, as computed by
  \texttt{scipy.\allowbreak{}sparse.\allowbreak{}linalg.\allowbreak{}eigen.\allowbreak{}eigs()}
  \todo{Cite ORIGEN and explain what this is (a pressurized water reactor with
    burnup of 50 MWd/kg)}.}
\label{fig:eigenvals-pwru50}
\end{figure}

\begin{figure}[!ht]
\centering
\includegraphics[width=0.9\textwidth]{eigenvals_decay.pdf}
\caption{Eigenvalues of the decay matrix, as computed by
  \texttt{scipy.\allowbreak{}sparse.\allowbreak{}linalg.\allowbreak{}eigen.\allowbreak{}eigs()}.}
\label{fig:eigenvals-decay}
\end{figure}
