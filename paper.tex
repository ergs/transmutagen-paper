\documentclass{article}

\usepackage{lmodern}
\usepackage[T1]{fontenc}
\usepackage[utf8x]{inputenc}
\usepackage[scaled=0.8]{DejaVuSansMono}
\bibliographystyle{plain}

% \usepackage[all]{xy}
\usepackage{amsmath}
\usepackage{caption}
% \graphicspath{ {images/} }

\usepackage{authblk}

% Makes quote characters in monospace font not be curly
\usepackage{upquote}

\usepackage{amsmath}
\usepackage{url}

% for nice units
\usepackage{siunitx}

% for images: png, pdf, etc
\usepackage{graphicx}

% for matplotlib pgf output
\usepackage{pgfplots}

% for nice table formatting, i.e., /toprule, /midrule, etc
\usepackage{booktabs}

% to allow for \verb++ declarations in captions.
\usepackage{cprotect}

% to allow usage of \mathbb symbols
\usepackage{amssymb}

\usepackage{longtable}

\usepackage{multirow}

\usepackage{listings}

\usepackage{algorithmic}
\usepackage{algorithm}
% Use C style comments
\renewcommand{\algorithmiccomment}[1]{{\color{gray} /* #1 */}}
% Don't put a colon after the line numbers
\algsetup{linenodelimiter=\ }
% Make line numbers smaller
\algsetup{linenosize=\tiny}
% Make \REQUIRE be "Input:"
\renewcommand{\algorithmicrequire}{\textbf{Input:}}
% Make \ENSURE be "Output:"
\renewcommand{\algorithmicensure}{\textbf{Output:}}
% Add \CONTINUE
\newcommand{\algorithmiccontinue}{\textbf{continue}}
\newcommand{\CONTINUE}{\STATE \algorithmiccontinue}

\usepackage{color}

\usepackage{ulem}

\usepackage{soul}

% Keep this last
\usepackage{hyperref}


% Clever trick from
% https://www.gijsk.com/blog/2008/06/absence-citation-needed/. Use
% \cite{ationneeded} to produce red "[citation needed]". You have to use
% \oldcite in things like footnotes or it will give you an error.
\usepackage{ifthen}
\let\oldcite=\cite
\renewcommand\cite[1]{\ifthenelse{\equal{#1}{ationneeded}}{{\color{red}[citation~needed]}}{\oldcite{#1}}}

\newcommand\todo[1]{{\color{red}TODO\@: #1}}

\newcommand\email[1]{\href{mailto:#1}{#1}}

\title{Transmutagen}

\author[1]{Aaron Meurer}%
\author[1]{Anthony Scopatz}%
\affil[1]{Department of Mechanical Engineering, University of South Carolina, Columbia, South Carolina, United States}%

\begin{document}
\flushbottom
\maketitle
\thispagestyle{empty}

\begin{abstract}

\end{abstract}

% TODO: Remove this
\tableofcontents

\section{Introduction}
\label{sec:intro}
Transmutagen is a Python library for generating fast nuclear transmutation
solvers using the Chebyshev Rational Approximation Method
(CRAM)~\cite{ationneeded} for the matrix exponential. Transmutagen depends on
Python 3.5 and SymPy 1.1 or newer \todo{Other dependencies?}. Transmutagen is
capable of computing the CRAM approximation digits for $e^{-t}$ on
$[0, \infty)$ for any degree, code generating such an approximation using
SciPy's linear algebra routines, and code generating a CRAM approximation into
custom fast C routine for a given sparsity pattern. Transmutagen also contains
utilities for testing the validity of the approximation, both through testing
of mathematical and physical properties, and through comparison of the results
to ORIGEN~\cite{ationneeded} and PyNE~\cite{ationneeded}.


\section{CRAM}
\label{sec:cram}
Consider the function $e^{-t}$ on the interval $[0,\infty)$. Let
$\pi_{k,k}\subset \mathbb{R}(t)$ be the set of rational functions $r_{k,k}(t)
= p_k(t)/q_k(t)$, where $p_k(t), q_k(t)\in \mathbb{R}[t]$ are polynomials of
degree at most $k$.

The Chebyshev Rational Approximation Method (CRAM)~\cite{ationneeded}
approximation of degree $k$ for $e^{-t}$ is the unique rational function
$\hat{r}_{k,k}(t)$ such that
\begin{equation}
  \sup_{t\in[0, \infty)}|\hat{r}_{k, k}(t) - e^{-t}|
\end{equation}
is minimized. Let
\begin{equation}
  \epsilon_{k,k} = \inf_{r_{k,k}\in\pi_{k,k}}\left\{\sup_{t\in[0, \infty)}|r_{k, k}(t) - e^{-t}|\right\}.
\end{equation}
$\epsilon_{k,k}$ is the absolute error of the approximation. It has been shown
that $\epsilon_{k,k} = O(H^{-k})$, where $H=9.289\ldots$~\cite{ationneeded} is
known as Halphen's constant.

\subsection{The Remez Algorithm}
\label{sec:remez-algorithm}
Transmutagen implements the Remez algorithm to compute the CRAM approximation.
All computations are performed using SymPy~\cite{10.7717/peerj-cs.103}
symbolic expressions with arbitrary precision floating point numbers, which
are backed by the mpmath library~\cite{ationneeded}.

To compute the CRAM approximation, the interval $[0, \infty)$ is first
translated to $[-1, 1)$ via the transformation $t\mapsto c\frac{t+1}{t-1}$.
\todo{cite Carpenter paper here.} \todo{discuss this {\it after} discussing
  finding the roots.} This is done so that the maxima of the error at each
stage are all on a bounded interval (see below). The constant $c$ is chosen so
that these are distributed sufficiently evenly across the interval $[-1, 1)$.
If they are clustered too much, the numeric root finding algorithm may fail to
find them all. We found that the value $c=k/\phi$, where $\phi=1.618\ldots$ is
the golden ratio, works well. The choice of $c$ does not affect the final
result on $[0, \infty)$, but deviations from this heuristic resulted in issues
in the root finding stage for large degrees.

To help visualize this, Figure~\ref{fig:cram-plot} shows the error in the
shifted approximation on $[-1, 1)$ and the final translated approximation on
$[0, \infty)$, for degree 14.

\begin{figure}[!ht]
\centering
\resizebox{\textwidth}{!}{\input{cram-plot.pgf}}
\caption{Left: plot of $\hat{r}_{14, 14}\left(c\frac{t+1}{t-1}\right) -
  e^{-c\frac{t+1}{t-1}}$; right: plot of $\hat{r}_{14, 14}(t) -
e^{-t}$ ($c=\frac{14}{\phi}\approx 8.652$).}
\label{fig:cram-plot}
\end{figure}

In transmutagen's implementation of the Remez algorithm, the expressions
\begin{equation}
  r := \frac{p_kt^k + \cdots + p_1t + p_0}{q_kt^k + \cdots +
    q_1t + 1},
\end{equation}
\begin{equation}
  D := e^{c\frac{t+1}{t-1}} - r,
\end{equation}
and
\begin{equation}
  E := D + (-1)^i\epsilon
\end{equation}
are represented symbolically as SymPy expressions. The expression $E$ is then
evaluated in $t$ at $2(k + 1)$ points in the interval $(-1, 1)$ for
$i=0\ldots 2k+1$. For the first iteration of the algorithm, any set of initial
points in $(-1, 1)$ can be used. By default, transmutagen uses the Chebyshev
nodes~\cite{ationneeded}, that is, the roots of $T_{2(k +1)}(x)$, which always
lie in the interval $(-1, 1)$.\footnote{$T_n(x)$ is the $n$th Chebyshev
  polynomial of the first kind, defined as $T_n(\cos(x)) = \cos(nx)$, e.g.,
  $T_2(x) = 2x^2 - 1$. The roots of $T_n(x)$ are
  $\cos{\left (\frac{2k - 1}{n}\frac{\pi}{2} \right )},\,k=1\ldots n$.} It can
also optionally use a random set of points. We found that convergence can take
as much as 50\% more steps when random initial points are used instead of the
Chebyshev nodes. This evaluation of $E$ results in a
nonlinear system of $2(k+1)$ equations in $2(k+1)$ variables
($p_0,\ldots,p_k,q_1,\ldots,q_k,\epsilon$). These are solved using the
\texttt{sympy.nsolve} function, which internally uses \texttt{mpmath.findroot}
to apply Newton's method~\cite{ationneeded} with a symbolically computed
Jacobian.

\label{sec:nsolve-bisection}
The solution to this system is then substituted into $D$, and the critical
points of this function on the interval $[-1, 1)$ are then found. This is done
by taking the symbolic derivative of $D$, splitting the interval into
subintervals, and finding the roots on each subinterval using the bisection
algorithm, via \texttt{nsolve}. Let $\{z_i\}$ be the set of critical points on
$[-1, 1)$, including the end-points $D|_{t=-1}$ and
$\lim_{t\to 1} D=-r|_{t=1}$. There are $2(k+1)$ points $\{z_i\}$. These points
$\{z_i\}$ are used as the set of initial points for the next iteration, and
the algorithm iterates as such until convergence is reached.

By the equioscillation theorem~\cite{ationneeded}, the approximation
$\hat{r}_{k, k}$ is minimal for a given order $k$ when the critical points of
$\hat{r}_{k, k}(t) - e^{-t}$ oscillate in sign and have equal absolute value.
For the iterates of the algorithm, the points $z_i$ are these critical points
of the current approximation, so convergence is detected by looking at
$\varepsilon_N = \max{|z_i|} - \min{|z_i|}$, for the $N$th iteration of the
algorithm. Convergence occurs roughly when $\varepsilon_N$ is near $10^{-d}$,
where $d$ decimal digits are used in the calculations. However, the true
minimal value of $\varepsilon_N$ depends on both $k$ and $d$. We found a
robust heuristic to be to iterate until $\log_{10}{(\varepsilon_N)}$ is near
$-d$, then stop iterating when the values of $\varepsilon_N$ become
log-convex. See Figure~\ref{fig:convergence-14-1000} for an example of the
convergence of $\varepsilon_N$ for degree 14, 1000 digits of precision.

\begin{figure}[!ht]
\centering
\resizebox{0.9\textwidth}{!}{\input{convergence-14-1000.pgf}}
\caption{Convergence of the Remez algorithm for degree 14, 1000 digits of
  precision.}
\label{fig:convergence-14-1000}
\end{figure}

Once the algorithm converges, the rational function is translated back to the
interval $[0, \infty)$ via the inverse transformation
$t\mapsto \frac{t - c}{t + c}$. This must be done with care to avoid losing
precision. For a polynomial $f$ and rational function $p/q$, the SymPy method
\texttt{Poly(f).\allowbreak{}transform(p, q)} efficiently computes
$q^nf\left(p/q\right)$ without losing precision. The resulting rational
function is normalized so that the constant term in the denominator is 1,
which matches the form used in the literature.~\cite{ationneeded}

The Remez algorithm is outlined in Algorithm~\ref{alg:remez-pseudocode} as pseudocode.
\begin{algorithm}
  \caption{The Remez algorithm for the CRAM approximation of $e^{-t}$ on
    $[0, \infty)$ of degree $k,k$.}\label{alg:remez-pseudocode}
  \begin{algorithmic}[1]
    \REQUIRE  $k$, the order of the approximation
    \STATE $t, p_0, \ldots, p_k, q_1, \ldots, q_k, \epsilon$ are symbolic variables
    \STATE $r \leftarrow \frac{p_kt^k + \cdots + p_1t + p_0}{q_kt^k + \cdots +
      q_1t + 1}$
    \STATE $c \leftarrow \frac{k}{\phi}$
    \STATE $D \leftarrow e^{c\frac{t+1}{t-1}} - r$

    \STATE $\{z_i\} \leftarrow$ Chebyshev nodes of order $2(k+1)$
    \STATE $N \leftarrow 1$
    \REPEAT
      \FOR {$i=1$ \TO $2(k+1)$}
        \STATE $E_i \leftarrow D|_{t=z_i} + (-1)^i\epsilon$
      \ENDFOR
      \STATE $sol \leftarrow$ Solve $\{E_i\}$ for $(p_0,\ldots,p_k,q_1,\ldots,q_k,\epsilon)$
      \STATE $z_0 \leftarrow D|_{sol,t=-1}$
      \STATE $z_i \leftarrow$ $i^\mathrm{th}$ critical point of $D|_{sol}$ on
      $(-1, 1)$ \COMMENT{$i=1\ldots 2n$}
      \STATE $z_{2(k + 1)} \leftarrow -r|_{sol,t=1}$ \COMMENT{$\lim_{t\to 1} D|_{sol}$}
      \STATE $\varepsilon_N \leftarrow \max{|z_i|} - \min{|z_i|}$
      \STATE $N \leftarrow N + 1$
    \UNTIL {convergence condition: $\log_{10}(\varepsilon_N) \approx -d$ \AND
      $\varepsilon_N$ is log-convex}
    \STATE $r_\mathrm{final}=r|_{t=\frac{t - c}{t + c}}$ \COMMENT{translate
      $[-1, 1)$ back to $[0, \infty)$ and normalize $q_0=1$}
    \ENSURE $r_\mathrm{final}$
  \end{algorithmic}
\end{algorithm}


\subsection{Results}
\label{sec:cram-results}
Transmutagen is able to compute the CRAM approximation for any degree $n$ to
any number of decimal digits $d$. We have computed the first $60$ degrees to
200 digits. Our digits agree with the coefficients published
in~\cite{carpenter1984extended}, which has the first 30 degrees up to 20
digits. It should be noted that the digits in~\cite{carpenter1984extended}
have been rounded using decimal rounding, which is often different from
floating point, or binary rounding. For example, the constant coefficient in
the numerator ($p_0$) for degree 11 is reported as
\texttt{1.0000000000146631119}, even though the true coefficient,
\texttt{1.000000000014663111949374871\ldots} is more closely approximated as a
binary floating point number by \texttt{1.0000000000146631120} \todo{Can we
find an example that matters when rounded to a \texttt{double}?}. Our computed
digits agree with those from~\cite{carpenter1984extended} when rounded using
the default half-even strategy of the Python \texttt{decimal} module (SymPy's
\texttt{Float} objects, which are used in the calculations from
Section~\ref{sec:remez-algorithm}, use binary rounding). This is important
because for practical purposes, when using the computed coefficients to
compute a CRAM approximation on a computer, one should round the digits to a
machine floating point number using binary rounding, to get the floating point
number that most closely approximates the true value. The difference is
typically no more than a single bit, but such errors in the approximation can
propagate to larger errors in the result, as shown in section
{\color{red}???}. \todo{Write something about that in another section.}


\section{Computing the Matrix Exponential from CRAM}
\label{sec:cram-matrices}
Given a square matrix $A$, the matrix exponential is defined
as~\cite{ationneeded}
\begin{equation}
  \label{eq:matrix-exponential}
  e^{A} = \sum_{n=0}^\infty \frac{A^n}{n!}.
\end{equation}

The CRAM approximation to $e^{-x}$ can be applied to approximate the matrix
exponential when the matrix $A$ has eigenvalues on or near the negative real
axis~\cite{pusa2010computing}. This is because the exponential of a matrix is
based on the exponentials of its eigenvalues.

% SymPy to compute Halphen's constant
% nsolve([1/s - exp(-pi*elliptic_k(1 - c**2)/elliptic_k(c**2)), elliptic_k(c**2) - 2*elliptic_e(c**2)], [c, s], [.9, 9])

The naive method of computing the matrix exponential $e^A$ via
$\hat{r}_{k, k}(t)=\frac{p_kt^k + \cdots + p_0}{q_kt^k + \cdots q_1t +
  1}=\frac{p}{q}$ is to compute
$\left .(q_kA^k + \cdots + q_1A + I)\middle\backslash(p_kA^k + \cdots + p_0 I)\right.$, where
$\backslash$ is a matrix solve. However, this method is very unstable, as the
coefficients $p_i,q_i$ are quite small (the smallest on the order of
$10^{-2k}$), and taking powers of $A$ is also unstable. We also attempted
using the Horner scheme on the numerator and denominator,
$\hat{r}_{k, k}(t)=\frac{p_0 + t(p_1 + t(p_2 + \cdots t(p_{k-1} + tp_k)))}{1 +
  t(q_1 + t(q_2 + \cdots t(q_{k-1} + tq_k)))}$, but found this to be
numerically unstable as well.

A better method is to perform a partial fraction decomposition on $\hat{r}_{k,
  k}(t)$~\cite{pusa2010computing}:
\begin{equation}
\label{eq:part-frac}
  \hat{r}_{k, k}(t) = \alpha_0 + \sum_{i=1}^k \frac{\alpha_i}{t - \theta_i},
\end{equation}
where $\theta_i$ are the roots of $q$, which are all
nonreal and have multiplicity 1, $\alpha_i$ is the residue of
$\hat{r}_{k, k}(t)$ at $t=\theta_i$, and $\alpha_0$ is the residue at
infinity. When $k$ is even, the roots $\theta_i$ all come in complex conjugate
pairs, so the sum can be reduced to
\begin{equation}
  \hat{r}_{k, k}(t) = \alpha_0 + \mathrm{Re}\left(\sum_{i=1}^{k/2}
    \frac{\alpha_i}{t - \theta_i}\right).
\end{equation}
Thus, the approximation to $e^Ab$ can be computed via
\begin{equation}
\label{eq:part-frac-matrix-re}
  \hat{r}_{k, k}(A)b = \alpha_0b + \mathrm{Re}\left(\sum_{i=1}^{k/2} \left. (A -
    \theta_i I)\middle\backslash(\alpha_i b) \right.\right).
\end{equation}

We attempted expand the real part of this expression via
\begin{equation}
\mathrm{Re}\left(\frac{\alpha_i}{t - \theta_i}\right) = \frac{\mathrm{Re}{(\alpha_{i})}t - \mathrm{Re}{(\alpha_{i})} \mathrm{Re}{(\theta_{i})} - \mathrm{Im}{(\alpha_{i})} \mathrm{Im}{(\theta_{i})}}{\left(t - \mathrm{Re}{(\theta_{i})}\right)^{2} + \mathrm{Im}{(\theta_{i})}^{2}},
\end{equation}
and compute
\begin{equation}
  % \hat{r}_{k, k}(A)b =
  \left.\left(\left(A - \mathrm{Re}{(\theta_{i})I}\right)^{2} +
  \mathrm{Im}{(\theta_{i})}^{2}I\right)\middle\backslash \left(\mathrm{Re}{(\alpha_{i})}A - (\mathrm{Re}{(\alpha_{i})} \mathrm{Re}{(\theta_{i})} + \mathrm{Im}{(\alpha_{i})} \mathrm{Im}{(\theta_{i})})b\right)\right..
\end{equation}
However, this was found to be numerically unstable, most likely because of the
additional matrix power. % TODO: verify this

In transmutagen roots $\theta_i$ are computed with
\texttt{sympy.\allowbreak{}nsolve} using Newton's method. \todo{Some
  discussion on the number of digits needed for the roots here.} The residues
$\alpha_i$ were computed from the standard formulas
$\alpha_i = \frac{p}{q/(t - \theta_i)}|_{t=\theta_i}$ and
$\alpha_0=\frac{p_k}{q_k}$.


\subsection{Choice of degree}
\label{sec:degrees}
In our tests, when using CRAM to compute transmutation with machine double
floating point arithmetic, the maximum absolute error was bounded at around
$10^{-14}$. Figure~\ref{fig:degrees} shows how the error levels off at around
degree 16 for a transmutation computation for smaller timesteps and around
degree 18 for larger timesteps. The error in general is higher for higher time
steps. This is because multiplying a matrix $A$ by $t$ scales the spectrum of
$A$ by $t$.

\begin{figure}[!ht]
\centering
\resizebox{0.9\textwidth}{!}{In our tests, when using CRAM to compute transmutation with machine double
floating point arithmetic, the maximum absolute error was bounded at around
$10^{-14}$. Figure~\ref{fig:degrees} shows how the error levels off at around
degree 16 for a transmutation computation for smaller timesteps and around
degree 18 for larger timesteps. The error in general is higher for higher time
steps. This is because multiplying a matrix $A$ by $t$ scales the spectrum of
$A$ by $t$.

\begin{figure}[!ht]
\centering
\resizebox{0.9\textwidth}{!}{In our tests, when using CRAM to compute transmutation with machine double
floating point arithmetic, the maximum absolute error was bounded at around
$10^{-14}$. Figure~\ref{fig:degrees} shows how the error levels off at around
degree 16 for a transmutation computation for smaller timesteps and around
degree 18 for larger timesteps. The error in general is higher for higher time
steps. This is because multiplying a matrix $A$ by $t$ scales the spectrum of
$A$ by $t$.

\begin{figure}[!ht]
\centering
\resizebox{0.9\textwidth}{!}{\input{degrees.pgf}}
\caption{Maximum absolute difference of computed values for degree $n$
  compared to $n-2$ for varying degrees and timesteps, using PWRU50 data with
  an initial vector $b$ representing a unit mass of U235.}
\label{fig:degrees}
\end{figure}

Figure~\ref{fig:error-plot} shows the error of the CRAM approximation in the
complex plane (note that it shows $\hat{r}(-t)$, so that it can be compared
against the spectrum of $A$). Comparing it with
Figures~\ref{fig:eigenvals-pwru50} and~\ref{fig:eigenvals-decay}, it can be
seen that CRAM should provide an accurate approximation for transmutation
matrices, but not for decay matrices.

\begin{figure}[!ht]
\centering
\includegraphics[width=0.9\textwidth]{error-plot.pdf}
\caption{Error of the approximation, $\left |\hat{r}_{14,14}(-t) - e^{t}\right
  |$ in the complex plane. }
\label{fig:error-plot}
\end{figure}

\begin{figure}[!ht]
\centering
\includegraphics[width=0.9\textwidth]{eigenvals_pwru50.pdf}
\caption{Eigenvalues of the transmutation matrix for pwru50, as computed by
  \texttt{scipy.\allowbreak{}sparse.\allowbreak{}linalg.\allowbreak{}eigen.\allowbreak{}eigs()}
  \todo{Cite ORIGEN and explain what this is (a pressurized water reactor with
    burnup of 50 MWd/kg)}.}
\label{fig:eigenvals-pwru50}
\end{figure}

\begin{figure}[!ht]
\centering
\includegraphics[width=0.9\textwidth]{eigenvals_decay.pdf}
\caption{Eigenvalues of the decay matrix, as computed by
  \texttt{scipy.\allowbreak{}sparse.\allowbreak{}linalg.\allowbreak{}eigen.\allowbreak{}eigs()}.}
\label{fig:eigenvals-decay}
\end{figure}
}
\caption{Maximum absolute difference of computed values for degree $n$
  compared to $n-2$ for varying degrees and timesteps, using PWRU50 data with
  an initial vector $b$ representing a unit mass of U235.}
\label{fig:degrees}
\end{figure}

Figure~\ref{fig:error-plot} shows the error of the CRAM approximation in the
complex plane (note that it shows $\hat{r}(-t)$, so that it can be compared
against the spectrum of $A$). Comparing it with
Figures~\ref{fig:eigenvals-pwru50} and~\ref{fig:eigenvals-decay}, it can be
seen that CRAM should provide an accurate approximation for transmutation
matrices, but not for decay matrices.

\begin{figure}[!ht]
\centering
\includegraphics[width=0.9\textwidth]{error-plot.pdf}
\caption{Error of the approximation, $\left |\hat{r}_{14,14}(-t) - e^{t}\right
  |$ in the complex plane. }
\label{fig:error-plot}
\end{figure}

\begin{figure}[!ht]
\centering
\includegraphics[width=0.9\textwidth]{eigenvals_pwru50.pdf}
\caption{Eigenvalues of the transmutation matrix for pwru50, as computed by
  \texttt{scipy.\allowbreak{}sparse.\allowbreak{}linalg.\allowbreak{}eigen.\allowbreak{}eigs()}
  \todo{Cite ORIGEN and explain what this is (a pressurized water reactor with
    burnup of 50 MWd/kg)}.}
\label{fig:eigenvals-pwru50}
\end{figure}

\begin{figure}[!ht]
\centering
\includegraphics[width=0.9\textwidth]{eigenvals_decay.pdf}
\caption{Eigenvalues of the decay matrix, as computed by
  \texttt{scipy.\allowbreak{}sparse.\allowbreak{}linalg.\allowbreak{}eigen.\allowbreak{}eigs()}.}
\label{fig:eigenvals-decay}
\end{figure}
}
\caption{Maximum absolute difference of computed values for degree $n$
  compared to $n-2$ for varying degrees and timesteps, using PWRU50 data with
  an initial vector $b$ representing a unit mass of U235.}
\label{fig:degrees}
\end{figure}

Figure~\ref{fig:error-plot} shows the error of the CRAM approximation in the
complex plane (note that it shows $\hat{r}(-t)$, so that it can be compared
against the spectrum of $A$). Comparing it with
Figures~\ref{fig:eigenvals-pwru50} and~\ref{fig:eigenvals-decay}, it can be
seen that CRAM should provide an accurate approximation for transmutation
matrices, but not for decay matrices.

\begin{figure}[!ht]
\centering
\includegraphics[width=0.9\textwidth]{error-plot.pdf}
\caption{Error of the approximation, $\left |\hat{r}_{14,14}(-t) - e^{t}\right
  |$ in the complex plane. }
\label{fig:error-plot}
\end{figure}

\begin{figure}[!ht]
\centering
\includegraphics[width=0.9\textwidth]{eigenvals_pwru50.pdf}
\caption{Eigenvalues of the transmutation matrix for pwru50, as computed by
  \texttt{scipy.\allowbreak{}sparse.\allowbreak{}linalg.\allowbreak{}eigen.\allowbreak{}eigs()}
  \todo{Cite ORIGEN and explain what this is (a pressurized water reactor with
    burnup of 50 MWd/kg)}.}
\label{fig:eigenvals-pwru50}
\end{figure}

\begin{figure}[!ht]
\centering
\includegraphics[width=0.9\textwidth]{eigenvals_decay.pdf}
\caption{Eigenvalues of the decay matrix, as computed by
  \texttt{scipy.\allowbreak{}sparse.\allowbreak{}linalg.\allowbreak{}eigen.\allowbreak{}eigs()}.}
\label{fig:eigenvals-decay}
\end{figure}


\section{Code generated solver}
\subsection{\texttt{sympy.lambdify}}
\label{sec:lambdify-solve}
Once the CRAM approximation of degree $k$, $\hat{r}_{k, k}(t)$ has been
computed, it can be used to compute the matrix exponential $e^{-A}$ by
substituting the matrix $A$ for $t$ in the expression. Transmutagen has a
Python class, \texttt{MatrixNumPyPrinter}, which subclasses and extends the
\texttt{sympy.printing.lambdarepr.NumPyPrinter} class, which is used by
\texttt{sympy.lambdify} to convert a SymPy expression to a string representing
an equivalent NumPy/SciPy expression. The \texttt{NumPyPrinter} class by
default deals with scalar expressions vectorized over arrays, and the
\texttt{MatrixNumPyPrinter} class extends it so that the expression operates
over matrices. In particular, the class ensures that

\begin{itemize}
\item multiplications use scalar multiplication (\texttt{*}) or
  matrix multiplication (\texttt{@}), as appropriate,
\item addition of a variable and a constant, such as $t + 2$, is represented
  by the addition of an identity matrix, $t + 2I$. Since the shape of the
  matrix is not known ahead of time, Transmutagen uses a special
  \texttt{autoeye} class which automatically evaluates as \texttt{scipy.sparse.linalg.eye} at runtime when the shape is known.
\item Divisions are represented by matrix solves, e.g., $\frac{t + 1}{t - 1}$
  would be represented by \texttt{solve(t - autoeye(1), t + autoeye(1))}.
\end{itemize}

The naive method of computing the matrix exponential via $\hat{r}_{k,
  k}(t)=\frac{p_kt^k + \cdots + p_0}{q_kt^k + \cdots q_1t + 1}$
is to compute $(q_kA^k + \cdots + q_1A + I)\backslash(p_kA^k + \cdots + I)$, where
$\backslash$ is a matrix solve. However, this method is very unstable, as the
coefficients $p_i,q_i$ are quite small (the smallest on the order of
$10^{-2k}$), and taking powers of $A$ is also unstable.

A better method is to perform a partial fraction decomposition on $\hat{r}_{k,
  k}(t)$:
\begin{equation}
  \hat{r}_{k, k}(t) = \alpha_0 + \sum_{i=1}^k \frac{\alpha_i}{t - \theta_i},
\end{equation}
where $\theta_i$ are the roots of $q_kt^k + \cdots + q_1t + 1$, which are all
nonreal and have multiplicity 1, $\alpha_i$ is the residue of
$\hat{r}_{k, k}(t)$ at $t=\theta_i$, and $\alpha_0$ is the residue at
infinity. When $k$ is even, the roots $\theta_i$ all come in complex conjugate
pairs, so the sum can be reduced to
\begin{equation}
  \hat{r}_{k, k}(t) = \alpha_0 + \mathrm{Re}\left(\sum_{i=1}^{k/2} \frac{\alpha_i}{t - \theta_i}\right),
\end{equation}
which is computed on a matrix $A$ via
\begin{equation}
  \hat{r}_{k, k}(A) = \alpha_0I + \mathrm{Re}\left(\sum_{i=1}^{k/2} (A -
    \theta_i I)\backslash(\alpha_i I) \right).
\end{equation}

The roots $\theta_i$ were computed with \texttt{sympy.nsolve} using Newton's
method.
% TODO: Some discussion on the number of digits needed for the roots here.

We attempted expand the real part of this expression via
\begin{equation}
\mathrm{Re}\left(\frac{\alpha_i}{t - \theta_i}\right) = \frac{\mathrm{Re}{(\alpha_{i})}t - \mathrm{Re}{(\alpha_{i})} \mathrm{Re}{(\theta_{i})} - \mathrm{Im}{(\alpha_{i})} \mathrm{Im}{(\theta_{i})}}{\left(t - \mathrm{Re}{(\theta_{i})}\right)^{2} + \mathrm{Im}{(\theta_{i})}^{2}},
\end{equation}
and compute
\begin{equation}
\left(\left(A - \mathrm{Re}{(\theta_{i})I}\right)^{2} +
  \mathrm{Im}{(\theta_{i})}^{2}I\right)\backslash \left(\mathrm{Re}{(\alpha_{i})}A - (\mathrm{Re}{(\alpha_{i})} \mathrm{Re}{(\theta_{i})} + \mathrm{Im}{(\alpha_{i})} \mathrm{Im}{(\theta_{i})})I\right).
\end{equation}
However, this was found to be numerically unstable, most likely because of the
additional matrix power. % TODO: verify this


\subsection{Precomputed LU Solve}
\label{sec:precomputed-lu-solve}
The CRAM method using Equation~\ref{eq:part-frac-matrix-re} involves solving
\begin{equation}
\label{eq:basic-matrix-solve}
 (A - \theta I)\backslash(\alpha b)
\end{equation}
$k/2$ times for fixed $A$ and $b$ and varying $\theta$ and $\alpha$. This
solve is typically done using the LU decomposition algorithm. For the
transmutation problem the $A$ matrix is a sparse matrix with a fixed sparsity
pattern: every entry in $A$ corresponds to a transmutation of one nuclide to
another, and only certain transmutations are ever physically possible.

Furthermore, in transmutation, $A$ consists only of real entries, whereas for
$A - \theta I$, $\theta$ is a nonreal complex number. Thus, the diagonal
entries of the matrices being solved are always nonzero. As a result, pivoting
is not necessary in the LU solve algorithm. This allows precomputing the exact
LU decomposition ahead of time without knowledge of the values of $A$, beyond
which are nonzero.

Pseudocode for the LU solve algorithm for solving $Mx=b$ without pivoting is
shown in Algorithm~\ref{alg:lu-pseudocode}.~\cite{ationneeded}
\begin{algorithm}[h]
  \caption{LU solve of $Mx=b$ without pivoting.}\label{alg:lu-pseudocode}
  \begin{algorithmic}[1]
  \REQUIRE $M_{n\times n}$, $b_{n\times 1}$
  \STATE \COMMENT{First, decompose $M=L\cdot U$.}
  \STATE \COMMENT{The lower triangular part of $LU$ will be $L - I$ and the upper triangular part will be $U$.}
  \STATE
  \STATE $LU \leftarrow \mathrm{copy}(M)$
  \STATE
  \FOR{$i=1$ \TO $n$}
      \FOR{$j=i+1$ \TO $n$}
          \STATE $LU_{j, i} \leftarrow LU_{j, i}/LU_{i, i}$\label{alg:lu-pseudocode-decompose-division}
          \FOR{$k=i+1$ \TO $n$}
              \STATE $LU_{j, k} \leftarrow LU_{j, k} - LU_{j, i}\cdot LU_{i, k}$\label{alg:lu-pseudocode-reduction-line}
          \ENDFOR
      \ENDFOR
  \ENDFOR
  \STATE
  \STATE \COMMENT{Now perform the solve.}
  \STATE $x \leftarrow \mathrm{copy}(b)$
  \STATE
  \STATE \COMMENT{Forward substitution}
  \FOR{$i=1$ \TO $n$}
      \FOR{$j=1$ \TO $i$}
          \STATE $x_i \leftarrow x_i - LU_{i, j}\cdot x_j$\label{alg:lu-pseudocode-forward-substitution-line}
      \ENDFOR
  \ENDFOR
  \STATE
  \STATE \COMMENT{Backward substitution}
  \FOR{$i=n$ \TO $1$}
      \FOR{$j=i+1$ \TO $n$}
          \STATE $x_i \leftarrow x_i -LU_{i, j}\cdot x_j$\label{alg:lu-pseudocode-backward-substitution-line}
      \ENDFOR
      \STATE $x_i \leftarrow x_i/LU_{i, i}$\label{alg:lu-pseudocode-solve-division}
  \ENDFOR
  \STATE
  \ENSURE $x_{n\times 1}$

\end{algorithmic}
\end{algorithm}
Note that many steps of Algorithm~\ref{alg:lu-pseudocode} can be removed if an
element of $LU$ is known to be 0. In particular,
$LU_{j, k} \leftarrow LU_{j, k} - LU_{j, i}\cdot LU_{i, k}$
(line~\ref{alg:lu-pseudocode-reduction-line}) becomes
$LU_{j, k} \leftarrow LU_{j, k}$, a no-op, if either $LU_{j, i}$ or
$LU_{i, k}$ are 0. Similarly, $x_i \leftarrow x_i -LU_{i, j}\cdot x_j$
(lines~\ref{alg:lu-pseudocode-forward-substitution-line}
and~\ref{alg:lu-pseudocode-backward-substitution-line}) becomes
$x_i \leftarrow x_i$ if $LU_{i,j}$ is 0.

Given a sparsity pattern for $M$, the elements of $LU$ start the same
as the elements of $M$, but more elements may become nonzero through the
application of line~\ref{alg:lu-pseudocode-reduction-line}. Starting with a
set of nonzero indices of $M$, $IJ=\{(i, j) | M_{i, j} \neq 0\}$, one can
recursively compute a set of nonzero indices for $LU$. The pseudocode for this
algorithm is shown in Algorithm~\ref{alg:make-ijk}. The idea is to mirror the
decomposition loop from Algorithm~\ref{alg:lu-pseudocode}, adding new index
pairs to the set of nonzero indices $IJK$ whenever both $(j, i)$ and $(i, k)$
are also in the set.

\begin{algorithm}[h]
  \caption{Generate the set of nonzero entries of $LU$ given a set of nonzero
    entries of $M_{n,n}$.}\label{alg:make-ijk}
  \begin{algorithmic}[1]
  \REQUIRE $IJ=\{(i, j) | M_{i, j} \neq 0\}$
  \STATE $IJK \leftarrow \mathrm{copy}(IJ)$
  \STATE
  \FOR{$i=1$ \TO $n$}
      \FOR{$j=i+1$ \TO $n$}
          \IF{$(j, i) \notin IJK$}
              \CONTINUE
          \ENDIF
          \STATE
          \FOR{$k=i+1$ \TO $n$}
              \IF{$(i, k) \in IJK$ \AND $(j, k) \notin IJK$}
                   \STATE Add $(j, k)$ to $IJK$
               \ENDIF
          \ENDFOR
      \ENDFOR
  \ENDFOR
  \ENSURE $IJK=\{(i,j)|LU_{i,j}\mathrm{\ is\ potentially\ nonzero\ in\ Algorithm~\ref{alg:lu-pseudocode}}\}$
\end{algorithmic}
\end{algorithm}

Critically, for the transmutation matrix, it \textit{is} known ahead of time
which entries of $A$ are nonzero. Thus, the steps above can be reduced or
eliminated for the entries that are known to be zero. As noted above, the
diagonal entries for solves used for the real transmutations matrices are
always nonzero. Thus, the divisions by $LU_{i,i}$ in
line~\ref{alg:lu-pseudocode-decompose-division} are always by a nonzero
number.

Additionally, the roots of the denominators of the CRAM approximations
($\theta_i$) do not correspond to the eigenvalues of the transmutation matrix,
so the solve of Equation~\ref{eq:basic-matrix-solve} is never degenerate, that
is, the divisions by $LU_{i,i}$ in the solve stage
(line~\ref{alg:lu-pseudocode-solve-division}) is never by zero.

Because no pivoting is done in the precomputed LU solve, the order of the
entries in the matrix is important for efficiency. The most obvious way to
order matrix entries is to order the nuclides by the charge of the nucleus
(atomic number) followed by atomic mass number and state number
(ZAS)~\cite{ationneeded}. However, a much more efficient order is to order by
atomic mass number followed by the charge of the nucleus and state number
(CINDER). The sparsity pattern generated from the PyNE data has 35256
nonzero entries, including the diagonals. Using the ZAS ordering, an
additional 20194 zero entries must be considered for the LU solve, whereas for
the CINDER ordering, only 14082 additional entries must be considered, a
reduction of 17.3\%. See figure~\ref{fig:lu-solve-ordering}.

\begin{figure}[!ht]
\centering
\resizebox{0.9\textwidth}{!}{\input{lu-solve-ordering.pgf}}
\caption{A subset of the transmutation matrix sparsity pattern from PyNE data
  with ZAS ordering (left) and Cinder ordering (Right).}
\label{fig:lu-solve-ordering}
\end{figure}


\subsection{C Generated Solver}
\label{sec:c-solve}
The CRAM method using equation~\ref{eq:part-frac-matrix-re} involves solving
\begin{equation}
\label{eq:basic-matrix-solve}
 (A - \theta I)\backslash(\alpha b)
\end{equation}
$k/2$ times for fixed $A$ and $b$ and varying $\theta$ and $\alpha$. For the
transmutation problem the $A$ matrix is a sparse matrix with a fixed sparsity
pattern: every row/column in $A$ corresponds to a transmutation of one nuclide
to another, and only certain transmutations are ever physically possible.

The LU solve without pivoting algorithm for solving a $Mx=b$ is shown in
pseudocode in Algorithm~\ref{alg:lu-pseudocode}.
\begin{algorithm}
  \caption{LU solve of $Mx=b$ without pivoting.}\label{alg:lu-pseudocode}
  \begin{algorithmic}[1]
  \REQUIRE $M_{n\times n}$, $b_{n\times 1}$
  \STATE \COMMENT{First, decompose $M=LU$.}
  \STATE \COMMENT{The lower triangular part of $LU$ will be $L - I$ and the upper triangular part will be $U$.}
  \STATE
  \STATE $LU \leftarrow \mathrm{copy}(M)$
  \STATE
  \FOR{$i=1$ \TO $N$}
      \FOR{$j=i+1$ \TO $N$}
          \STATE $LU_{j, i} \leftarrow LU_{j, i}/LU_{i, i}$\label{alg:lu-pseudocode-decompose-division}
          \FOR{$k=i+1$ \TO $N$}
              \STATE $LU_{j, k} \leftarrow LU_{j, k} - LU_{j, i}\cdot LU_{i, k}$\label{alg:lu-pseudocode-example-line}
          \ENDFOR
      \ENDFOR
  \ENDFOR
  \STATE
  \STATE \COMMENT{Now perform the solve.}
  \STATE $x \leftarrow \mathrm{copy}(b)$
  \STATE
  \STATE \COMMENT{Forward substitution}
  \FOR{$i=1$ \TO $N$}
      \FOR{$j=1$ \TO $i$}
          \STATE $x_i \leftarrow x_i - LU_{i, j}\cdot x_j$
      \ENDFOR
  \ENDFOR
  \STATE
  \STATE \COMMENT{Backward substitution}
  \FOR{$i=N$ \TO $1$}
      \FOR{$j=i+1$ \TO $N$}
          \STATE $x_i \leftarrow x_i -LU_{i, j}\cdot x_j$
      \ENDFOR
      \STATE $x_i \leftarrow x_i/LU_{i, i}$\label{alg:lu-pseudocode-solve-division}
  \ENDFOR
  \STATE
  \ENSURE $x_{n\times 1}$

\end{algorithmic}
\end{algorithm}

Note that many steps of the LU algorithm can be removed if an element of $M$
(i.e., $A - \theta I$) is known to be 0. For example,
$LU_{j, k} \leftarrow LU_{j, k} - LU_{j, i}\cdot LU_{i, k}$
(line~\ref{alg:lu-pseudocode-example-line}) becomes
$LU_{j, k} \leftarrow LU_{j, k}$, a no-op, if either $LU_{j, i}$ or $LU_{i, k}$
are 0. If an element of $M$ off the diagonal, say $M_{a,b}$ is known to be
0, then recursively, \todo{Finish argument. Add pseudocode for \texttt{make\_ijk()}.}

Critically, for the transmutation matrix, it \textit{is} known ahead of time
which entries of $A$ are nonzero. Thus, the steps above can be reduced or
eliminated for the entries that are known to be zero. Furthermore, for the
transmutation problem, $A$ consists only of real entries, whereas for
$A - \theta I$, $\theta$ is a nonreal complex number. So the diagonal entries
are not zero. Thus, the division by $LU_{i,i}$ in
line~\ref{alg:lu-pseudocode-decompose-division} is always by a nonzero number.
In other words, pivoting is not necessary, which allows precomputing the exact
decomposition ahead of time without knowledge of the values beyond which are
nonzero.

Additionally, the roots of the denominators of the CRAM approximations do not
correspond to the eigenvalues of the transmutation matrix, so the solve of
Equation~\ref{eq:basic-matrix-solve} is never degenerate, that is, the
division by $LU_{i,i}$ in the solve stage
(line~\ref{alg:lu-pseudocode-solve-division}) is never by zero.

The command \texttt{python -m transmutagen.gensolve} takes a given sparsity
pattern for $A$ and generates a C function that solves
$(A - \theta)x =\alpha b$ (a default sparsity pattern based on data from
PyNE~\cite{ationneeded} is included with Transmutagen\todo{Discuss separate sparsity
  pattern for ORIGEN nuclides}). Additional C functions
are generated from the CRAM approximations of given orders (by default, 6, 8,
10, 12, 14, 16, and 18, but any even order can be used), which compute
$e^{-A}b$.

The command generates a C source file and header from a
Jinja~\cite{ationneeded} template based on the pseudocode in
Algorithm~\ref{alg:lu-pseudocode}. The source uses C99 complex numbers for the
arithemetic. Each line that is known to be a no-op from the provided sparsity
pattern is automatically removed. The resulting C source file is 12 MB with
the default orders \todo{Update this for latest sparsity pattern}. Compilation
of this file requires disabling most optimizations, as otherwise the compiler
either does not finish or runs out of memory. However, certain compilation
flags were found to speed up the performance of the algorithm, particularly
flags to speed up complex number operations. By default, complex numbers in C
are slow due to NaN checks, but these can be disabled to make the code
faster. \todo{Better justification on why these can be disabled.} Through
experimentation, we found the GCC flags \texttt{-O0 -fcx-fortran-rules
  -fcx-limited-range -ftree-sra -ftree-ter -fexpensive-optimizations} provided
speedups without adversely slowing down compile times. For Clang, we found the
flags \texttt{-O0 --ffast-math}. \todo{Should we discuss Clang here, such as
  how it is slower and doesn't seem to optimize complex numbers?}
\todo{Some performance information here (compile times; timing with/without
  optimizations). See also Section~\ref{sec:origen-speed}.}

To make the method more accessible to nuclear scientists, sparsity pattern is
generalized as a list of nuclides and a list of transitions between nuclides
(from--to pairs) that may be represented in the input matrix $A$. The
performance of this method is outlined in section~\ref{sec:origen-speed}.


\section{Comparison to Pusa~\oldcite{pusa2012correction}}
\label{sec:pusa-comparison}
\todo{This section requires math from section~\ref{sec:matrix-cram} below. We
  should probalby have a section on the math of partial fraction, which also
  outlines how we compute it.}

In addition, we have compared our partial fraction digits to those published
in~\cite{pusa2012correction} for degrees 14 and 16. However, many
discrepancies were found. These discrepancies are summarized in {\color{red}
  ???}.

Two values, $\mathrm{re}(\alpha_4)$ and $\mathrm{re}(\alpha_6)$ of degree 16,
differed within machine floating point precision (each within one-bit). The
values cause the approximations to differ by as much as $6\times10^{-19}$ (see
Figure~\ref{fig:pusa-differences}).

\begin{figure}[!ht]
\centering
\resizebox{0.9\textwidth}{!}{\input{pusa-differences.pgf}}
\caption{Difference between $\hat{r}_{k,k}$ for our computed values and the
  values from~\oldcite{pusa2012correction} for $k=14,16$.}
\label{fig:pusa-differences}
\end{figure}

From equation~\ref{eq:part-frac}, we can see that
$\lim_{t\to\infty}{\left|\hat{r}_{k,k} - e^{-t}\right|} = \alpha_0$. By the
equioscillation theorem \todo{A plot of the approximation to reference here
  would be helpful}, the error of the approximation equioscillates $2k$ times
in $(0, \infty)$, and additionally takes on the same value at $\infty$.\footnote{Recall
from the Remez algorithm above that on the translated interval, there are $2k$
points $z_i$, which represent the error of the approximation, and
$z_{2(k+1)}=\lim_{t\to 1}{D}$. When the approximation is translated from $[-1,
1)$ to $[0, \infty)$, the point $z_{2(k + 1)}$ represents the ``error at
$\infty$''.} What this means is that the computed value for $\alpha_0$ should
correspond to the maximum absolute error of the approximation.

This provides us with a method to test the validity of partial fraction
coefficients. If a value of expression $E(t) = \hat{r}_{k,k}(t) - e^{-t}$
\todo{Better variable name than $E$, to avoid confusion with algorithm above?}
exceeds $\alpha_0$ in absolute value, the coefficients are inconsistent.
Conversely, if the critical points of $E$, which can be found numerically via
gradient descent or by numerically solving the symbolic derivative, correspond
to $\alpha_0$, we can have a degree of confidence that the coefficients are
correct.

The different coefficients are shown in Tables~\ref{table:pusa-degree-14}
and~\ref{table:pusa-degree-16}. \input{pusa-table.tex}

\todo{Perform this analysis and report the results.}


\section{Sanity checks}
\label{sec:sanity-checks}
\todo{Anthony will need to review this text, or rewrite it if necessary.}
In addition to numeric tests, it is possible to run a sanity check on the CRAM
method based on physical tests. If $A$ is a transmutation matrix with no
fission, $e^{-At}$ represents {\color{red}\ldots}. The $i$th column of
$e^{-At}=e^{-At}I$ is equal to $e^{-At}\delta_{i}$ where $\delta_i$ is a
column vector with $1$ in the $i$th row and $0$ elsewhere. That is, the $i$th
column of $e^{-At}$ represents the transmutation of a unit mass of the $i$th
nuclide after $t$ seconds with no fission. By the conservation of mass, these
columns should each sum to 1, the mass that was started. While pure
transmutation of an isolated unit mass of each nuclide may not be physically
feasible, this provides a mathematical sanity check for different methods of
computing $e^{-At}$.

Mathematically, $A$ is a matrix where each non-diagonal entry is nonnegative,
and each diagonal entry is the negative of the sum of the non-diagonal entries
in its column, so that the columns of $A$ each sum to $0$. Note that if $A$ is
a matrix with the property that every column sums to $0$, i.e.,
$[1 \cdots 1] A = [0 \cdots 0]$ \todo{is there a name for these matrices?},
then so is $A^n$ for every $n\geq 1$. This is because
$[1 \cdots 1] A = [0 \cdots 0]$, so
$[1 \ldots 1] A^n = [0 \cdots 0] A^{n-1} = [0\cdots 0]$.

Consequently, using equation~\ref{eq:matrix-exponential},
$[1 \cdots 1]e^{-At} = [1 \cdots 1] (I + -At + \frac{(-At)^2}{2} + \cdots) = [1\ldots
1] + [0 \ldots 0] + [0 \ldots 0] + \cdots = [1\ldots 1]$. That is,
the columns of $e^{-At}$ each sum to $1$.

Figures~\ref{fig:nofission-pwru50-1-day}, \ref{fig:nofission-pwru50-1-year},
\ref{fig:nofission-pwru50-1000-years},
and~\ref{fig:nofission-pwru50-1-million-years} show the results of this sanity
check for various solvers and time steps. The matrix $A$ was generated from
the ORIGEN data library \texttt{pwru50.lib}, with fission product yields
omitted so that its columns sum to $0$. The figures show histograms of the
column sums of $e^{-At}$ minus 1 for $t$ equal to 1 day, 1 year, 1000 years,
and 1 million years. A perfect calculation would have every column sum to
exactly zero.

The leftmost histograms show
\texttt{scipy.\allowbreak{}sparse.\allowbreak{}linalg.\allowbreak{}expm} for
comparison. The remaining show CRAM with degree 14 using
\texttt{sympy.\allowbreak{}lambdify} using
\texttt{scipy.\allowbreak{}sparse.\allowbreak{}linalg}, with both the UMFPACK
and SuperLU backends, and a C solver generated by transmutagen against the
sparsity pattern of $A$. Note that for CRAM with degree 14, the absolute error
is expected to be on the order of $10^{-14}$. These figures show that CRAM is
highly accurate even for large time steps. The \texttt{expm} implementation in
\texttt{scipy.\allowbreak{}sparse} is inaccurate even for small time steps,
and is extremely inaccurate for large time steps.
\texttt{scipy.\allowbreak{}sparse.\allowbreak{}linalg.\allowbreak{}expm} gave
a resulting matrix with NaN values for $t = 1 \mathrm{\ million\ years}$.
Finally, UMFPACK has issues with a handful of nuclides at small timesteps and
with most nuclides at 1 million years, whereas SuperLU and the transmutagen
generated C solver have nearly identical precision characteristics, staying
accurate even at large time steps.

\begin{figure}[!ht]
\centering
\resizebox{0.9\textwidth}{!}{\input{nofission-pwru50-1-day.pgf}}
\caption{Sanity check for 1 day.}
\label{fig:nofission-pwru50-1-day}
\end{figure}

\begin{figure}[!ht]
\centering
\resizebox{0.9\textwidth}{!}{\input{nofission-pwru50-1-year.pgf}}
\caption{Sanity check for 1 year.}
\label{fig:nofission-pwru50-1-year}
\end{figure}

\begin{figure}[!ht]
\centering
\resizebox{0.9\textwidth}{!}{\input{nofission-pwru50-1000-years.pgf}}
\caption{Sanity check for 1000 years.}
\label{fig:nofission-pwru50-1000-years}
\end{figure}

\begin{figure}[!ht]
\centering
\resizebox{0.9\textwidth}{!}{\input{nofission-pwru50-1-million-years.pgf}}
\caption{Sanity check for 1 million years.}
\label{fig:nofission-pwru50-1-million-years}
\end{figure}


\section{ORIGEN Comparison}
\todo{Anthony}
\label{sec:origen-comparison}
\begin{itemize}
\item \it{Mass fraction vs. atom fraction}
\item {\it Same data}
\item $\alpha$ as He4
\item Fission yields for actinides
\item Cf248 not produced in 1 Month brwu, brwus, pwrdu3th, pwrputh, pwru,
  pwru50, pwrue, pwrus
\item More differences for larger time steps than 100 days
\item The C solver and SuperLU agree to within 1e-12
\item UMFPACK agrees to within 1e-8
\item \todo{Anthony will need to write this. See ORIGEN run logs in Google Drive}
\end{itemize}

As a more rigorous sanity check, the ORIGEN 2.2~\cite{ationneeded} library was
compared against, for several time steps and starting nuclides. ORIGEN 2.2 was
run against 22 ORIGEN libraries \todo{list the libraries?}, 7 time steps
($t= 1$ second, 1 day, 1 month, 1 year, 10 years, 1000 years, and 1 million
years), and 7 starting nuclides (Th232, U233, U235, U238, Pu239, Pu241, Cm245,
Cf249), and the results were compared against CRAM using
\texttt{sympy.\allowbreak{}lambdify} using UMFPACK,
\texttt{sympy.\allowbreak{}lambdify} using SuperLU, and a C solver generated
by transmutagen against the ORIGEN libraries' sparsity pattern.

\todo{This is just describing Meeseeks for now.} The speed of these analyses is summarized in
figures~\ref{fig:origen-scopatz},~\ref{fig:origen-aaron},
and~\ref{fig:origen-meeseeks}. The performance of ORIGEN is dependent on the
time step being computed. For small time steps (1 second to 1 year) ORIGEN
takes from 0.5\;s to 1\;s to run. At larger time steps, it takes increasingly
long---as long as 1110\;s at 1 million years. CRAM on the other hand, has a
runtime that is independent of the timestep. UMFPACK runs from 130\;ms to
280\;ms, and SuperLU runs from 80\;ms to 150\;ms. The generated C solver runs
from 30\;ms to 50\;ms. \todo{Should we report means or anything like that?}

In addition to performance comparison, the runs were also used to compare the
output of the four solvers. A large number of discrepancies were found for the
larger timesteps. However, according to the ORIGEN manual, ORIGEN is only
designed to give accurate values for time steps less than 100
days~\cite{ationneeded}.

All comparisons against the computed values were done with an absolute
tolerance of $10^{-5}$ and a relative tolerance of $10^{-3}$.\footnote{Using
  \texttt{numpy.\allowbreak{}allclose}, which compares $|actual -
  desired|$ to $atol + rtol|desired|$,
  where $actual$ is the value computed by CRAM and $desired$ is the
  value computed by ORIGEN.} The relative tolerance was chosen because ORIGEN
produces 4 digits of output. The absolute tolerance was chosen ORIGEN computes
the matrix exponential $e^{-A}$ using a Taylor expansion method using the error term
$\frac{e^{\mathrm{ASUM}}\mathrm{ASUM}^n}{n!}$,\footnote{ORIGEN computes this using
    Sterling's approximation, i.e.,
    $\frac{e^{\mathrm{ASUM}}(\frac{\mathrm{ASUM}e}{n})^n}{\sqrt{2\pi
      n}}$. c.f. lines 5075-5100 of the ORIGEN 2.2 source code.} where
$\mathrm{ASUM}$ is the maximum of the column sums of the matrix $A$ and $n =
3.5\mathrm{ASUM} + 6$. Due to fission,\todo{Write more here} the maximum
column sum is $\approx 2$, giving $\frac{e^{2}2^{13}}{13!}\approx 10^{-5}$.

For the 1 second, 1 day, and 1 month time steps,


\todo{Which of Figures~\ref{fig:origen-scopatz},~\ref{fig:origen-aaron},~\ref{fig:origen-meeseeks} should we show here?}

\todo{Pull in exact max/min data}
\begin{figure}[!ht]
\centering
\resizebox{0.9\textwidth}{!}{\input{origen-scopatz.pgf}}
\caption{Runtimes for different solvers computing transmutation over several starting libraries, nuclides, and timesteps.
(Scopatz machine)}
\label{fig:origen-scopatz}
\end{figure}

% Aaron's Mac: Mid 2012 MacbookPro, 2.3 GHz Intel Core i7, 8 GB 1600 MHz DDR3
\begin{figure}[!ht]
\centering
\resizebox{0.9\textwidth}{!}{\input{origen-aaron.pgf}}
\caption{Runtimes for different solvers computing transmutation over several starting libraries, nuclides, and timesteps.
 (Aaron machine)}
\label{fig:origen-aaron}
\end{figure}

\begin{figure}[!ht]
\centering
\resizebox{0.9\textwidth}{!}{\input{origen-meeseeks.pgf}}
\caption{Runtimes for different solvers computing transmutation over several starting libraries, nuclides, and timesteps.
 (Meeseeks)}
\label{fig:origen-meeseeks}
\end{figure}


\section{Pyne decay comparison}
\todo{Anthony}
\label{sec:pyne-decay-comparison}

\clearpage
\bibliography{paper}

\end{document}
